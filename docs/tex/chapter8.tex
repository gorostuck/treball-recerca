\newgeometry{margin=3cm, hmargin=2cm}
\chapter{Valoración general y conclusión}
\section{Procesos por los que pasó el Treball de Recerca}
Inicialmente el objetivo del Treball de Recerca era el de crear una escena tridimiensional interactiva en la que se mostrase alguna figura. Tras unas pocas semanas, nos dimos cuenta de que hoy en día existen muchas herramientas para realizar esto, pero nos daba la sensación de que no se comprendía muy bien cómo funcionaba, y por lo tanto no era honesto simplemente utilizar esas herramientas, duramente se podría extraer una conclusión.

Por ello decidimos ir un paso más allá e intentar crear nosotros mismos esas herramientas, es decir, crear una librería gráfica con nuestras propias funciones que asimilaran lo que hacían las librerías gráficas populares. Sin embargo hicimos un poco de trampa, pues es increíblemente difícil crear un software que se comunique propiamente con una tarjeta gráfica moderna (hoy en día vienen bloqueadas y se requiere contactar con el fabricante para tener el permiso). Además de la dificultad del proceso, es importante recordar que el enfoque que pretendíamos dar a este trabajo era matemático y no técnico o tecnológico, por lo que entrar en detalle sobre el funcionamiento de una tarjeta gráfica moderna se escapaba de las intenciones originales y no es representativo de una situación moderna en un equipo de desarrollo de Software.

Por lo tanto, nuestra \textit{librería gráfica} utilizaba OpenGL en el fondo, pero éste sólo se dedicaba a comunicarse con la tarjeta, y la manipulación los vértices, el cálculo operaciones y la decisión de qué se muestra en la pantalla quedaron relegados a nuestra librería gráfica.

Aunque parezca que el trabajo se ha simplificado mucho, aún seguía siendo una tarea mucho más laboriosa que el acercamiento original. Esta nueva dirección nos permite atajar algo tan complejo como el mundo de los motores gráficos con preguntas relativamente básicas cómo:
\begin{itemize}
\item{¿Cómo se hace una línea?}
\item{¿Qué operaciones se tienen que aplicar para pasar de coordenadas imaginarias las coordenadas concretas de una ventana?}
\item{¿Cómo funciona la ocultación de caras?}
\item{¿Cómo se calcula la rotación de un punto en tres dimensiones?}
\end{itemize}

Y estas son sólo unas pocas de las docenas que han surgido a lo largo del Treball de Recerca y se han ido respondido de forma bastante satisfactoria. Claro está, los recursos son los que hay y han habido cuestiones que han sido más difíciles y prácticamente imposibles de resolver, pero que al menos han surgido en primer momento.


\newpage
Finalmente, coincidiendo con la aproximación de la fecha de entrega del documento redactado, tocó el momento de trazar la línea y intentar pasar a centarnos en cómo enfocar la defensa del Treball de Recerca. Para esta, al no ser importante el conocimiento adquirido, usaremos herramientas que nos permitan comunicar y explicar el trabajo de una forma más directa y comprensible.

\section{Revisión de la hipótesis}
La hipótesis inicial era la siguiente:

\textit{Los equipos de desarrollo de software deberían de aspirar a ser ellos mismos quienes crean las librerías que utilizará el programa en sí. En caso de que esta sea una decisión no rentable, su máxima prioridad debería de ser entender cómo funcionan las herramientas y librerías que utilizarán.}

Aunque quizás puede parecer que este trabajo sea más una compilación de cómo funciona una librería gráfica desde un punto matemático más que un intento de comprobar esta hipótesis, yo era plenamente consciente de esto a la hora de formularla. Pensé que ser capaz de crear algo similar a una librería gráfica debería de proporcionarme el conocimiento para determinar si alguien con un cierto conocimiento previo sobre ordenadores y programación necesita adentrarse plenamente en los fundamentos de un motor gráfico para poder llegar a usarlo correctamente.

La evidencia de que este parece ser el caso se encuentra en los commits\footnote{actualización que hace un programador del código de un repositorio} que han ido ocurriendo a lo largo de la elaboración del código que acompaña este treball de recerca. Es muy fácil apreciar que en cuestión de semanas se pasa de utilizar operaciones ineficientes, guardar información inútil y en general hacer un mal uso de las funciones creadas a utilizar matrices y punteros para optimizar la velocidad y incluso aportar un cierto nivel de modularidad.

Por lo tanto, me parece justo establecer que el mundo de los gráficos computables requiere que el programador sea altamente consciente del funcionamiento interno del mismo para poder hacer un uso óptimo.

La prueba definitiva de que mi hipótesis está en lo ciertose halla en la evolución de OpenGL como librería gráfica. En las versiones originales de OpenGL (1 y 2), la misma librería gráfica, además de gestionar la comunicación con la tarjeta gráfica, hace todos los cálcuos y gestiona los vértices. Sin embargo, en la versión más reciente, OpenGL ha quedado completamente relegado a comunicación directa con la tarjeta gráfica, y todas las funciones de más (justamente las mismas que hemos creado nosotros en nuestra librería gráfica) han creado deprecadas y ahora es tarea del equipo de desarrollo que usa OpenGL crear las suyas propias, pensadas para funcionar de forma óptima en ese proyecto, pues un motor, o incluso librería, con uso general, parece ser cada vez más inviable.

\section{Opinión personal}
Estoy muy satisfecho con el resultado de este Treball de Recerca. Aunque he intentado dejar este documento lo más teórico posible, la mayoría de horas se han invertido en progaramar el motor gráfico utilizando la teoría que se ha expuesto. Espero poder continuar investigando sobre el complejo mundo de los motores gráficos más adelante, pero ciertamente este primer trabajo ha servido para tener una (limitada, pero no negligible) idea de cómo funciona la investigación.

\section{Bibliografía}
Diversos de estos recursos se han ido usando simultáneamente a lo largo del Treball de Recerca
\begin{itemize}
\item{Wiki de SDL2 \\ \url{https://wiki.libsdl.org}}
\item{Página web de Song Ho Ahn sobre OpenGL \\\url{https://www.songho.ca/opengl/index.html}}
\item{Página web de referencia oficial sobre OpenGL de Khronos \\\url{https://www.khronos.org/registry/OpenGL-Refpages/}}
\item{Wiki de \LaTeX \\ \url{https://en.wikibooks.org/wiki/LaTeX}}
\item{El repositorio del Treball de Recerca creado por mi, crucial para la comprensión del mismo\\
  \url{https://github.com/gorostuck/treball-recerca}}
\end{itemize}
