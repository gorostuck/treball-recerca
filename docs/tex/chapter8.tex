\chapter{Espacio de la pantalla y escalado y translación}
\section{Conversión a coordenadas de la pantalla}
Una vez se tienen las coordenadas de los puntos entre las coordenadas (-1, -1) y (1, 1) se busca encontrar la coordenada correspondiente en la pantalla.
\begin{itemize}
  \item{\(x_d\) es la coordenada en la pantalla}
  \item{\(x_s\) es la coordenada real normalizada}  
  \item{\(x_m\) es la coordenada donde comienza la superfície en la que queremos dibujar}
  \item{\(x_M\) es la coordenada donde termina la superfície en la que queremos dibujar}  
  \item{\(\Delta x = x_M - x_m\)}
\end{itemize}
\begin{equation*}
  \frac{x_s-(-1)}{1-(-1)} = \frac{x_d - x_m}{\Delta x}
\end{equation*}
Si aislamos \(x_d\) obtenemos la siguiente ecuación:
\begin{equation*}
  x_d = \frac{\Delta x \cdot x_s}{2}+\frac{\Delta x}{2}+x_m
\end{equation*}
En el caso de la coordenada \(y\) funciona igual excepto que cambiamos el sentido así que tiene una pequeña variación.
\begin{equation*}
  y_d = \frac{-(\Delta y) \cdot y_s}{2}+\frac{\Delta y}{2}+y_m
\end{equation*}
Y en forma de matriz, la operación quedaría así.

\begin{equation*}
  \begin{bmatrix}
     \frac{\Delta x}{2} && 0 && \frac{\Delta x}{2} + x_m\\
     0 && - \frac{\Delta y}{2} && \frac{\Delta y}{2} + y_m \\
     0 && 0 && 1
  \end{bmatrix}
  \cdot
  \begin{bmatrix}
    x_s\\
    y_s\\
    1
  \end{bmatrix}
  =
  \begin{bmatrix}
    x_d\\
    y_d\\
    1
  \end{bmatrix}
\end{equation*}

