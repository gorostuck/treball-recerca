\chapter{Espacio local del objeto}
En esta sección comentaré como funciona el sistema de gestión dinámico de vértices y el dibujo de la estructura que forman.
\section{Nodos}
Un nodo es una estructura básica multiprósito que además de contener valores contiene punteros hacia otros nodos.
\subsection{Nodo en el código}
\begin{lstlisting}[language=C]
  struct Node {
    GLenum Type;          // Tipo de nodo
    GLfloat Real[4];      // Matriz coordenadas reales
    GLfloat Projected[4]; // Matriz coordenadas proyectadas
    GLfloat Screen[3];    // Matriz coordenadas pantalla
    GLfloat Color[4];     // Matriz con los colores
    struct Node *next;    // Puntero hacia el nodo a la derecha  
    struct Node *inf;     // Punerto hacia el nodo debajo
  };
\end{lstlisting}
\subsection{Tipos de nodos}
Básicamente podemos distinguir entre los diferentes tipos de nodos dependiendo de como deben de ser interpretados.
\begin{itemize}
\item{\textbf{Inicio} : es el primer nodo, no tiene ninguna información excepto que señala al primer nodo con información útil.}
\item{\textbf{Encabezado} : en el tipo contiene la forma en la que se tienen que unir los vértices que se encuentran debajo de este nodo, a la derecha tiene otro encabezado o nada si es el último encabezado.}
  \item{\textbf{Vértice} : contiene información sobre las coordenadas reales, proyectadas y en la pantalla, además del color. Debajo tiene otro vértice excepto en caso de ser el último vértice de esa estructura.}
\end{itemize}
\section{Organización de la información dentro de las matrices del nodo}
En el nodo de un vértice se puede encontrar la información del mismo en las siguientes matrices.
\begin{figure}[ht]
  \[
  \centering
  \begin{bmatrix}
    x_{real} \\
    y_{real} \\
    z_{real}  \\
    1 
  \end{bmatrix}
  \begin{bmatrix}
    x_{proyectada} \\
    y_{proyectada} \\
    z_{proyectada} \\
    1
  \end{bmatrix}
  \begin{bmatrix}
    x_{pantalla} \\
    y_{pantalla} \\
    1
  \end{bmatrix}
  \begin{bmatrix}
    color_r \\
    color_g \\
    color_b \\
    color_a
  \end{bmatrix}
  \]
  \caption{Matriz de las coordenadas reales, las coordenads proyectadas, las coordenadas en la pantalla y el color.}
\end{figure}
\section{Sintaxis de la gestión de vértices}
Para ir creando vértices se tiene que estructurar el código de forma similar a este ejemplo.
\begin{lstlisting}[language=C]
  glColor3f(0.0f, 0.0f, 0.0f);
  glBegin(GL_QUADS);
  glVertex3f(0.75f, 0.75f);
  glVertex3f(0.75f, -0.75f);
  glVertex3f(-0.75f, -0.75f);
  glVertex3f(-0.75f, 0.75f);
  glEnd();
\end{lstlisting}
Se explicará el funcionamiento de esas funciones en las siguientes subsecciones.
\section{Colores en los vértices}
Utilizando la función
\begin{lstlisting}[language=C]
  glColor3f(Glclampf red, GLclampf green, GLclampf blue)
\end{lstlisting}
se puede asignar un color a todos los vértices que se definan posteriormente hasta que se vuelva a llamar una función de color.
\section{Sistemas de unión de vértices}
Para notificar al programa de que queremos una nueva estructura gráfica llamamos a la función
\begin{lstlisting}[language=C]
  glBegin(GLEnum mode)
\end{lstlisting}
Y como único argumento introducimos el modo de dibujo que queremos utilizar. Los distintos modos que se ofrecen son los siguientes.
\subsection{GL\_QUAD}
El número de vértices para el uso correcto debe de ser divisor de 4. Se ejecutarán los siguientes pasos.
\begin{itemize}
  \item{Línea del vértice 0 al vértice 1}
  \item{Línea del vértice 1 al vértice 2}
  \item{Línea del vértice 2 al vértice 3}
  \item{Línea del vértice 3 al vértice 0}
    \item{Se tomarán los siguientes cuatro vértices en caso de que existan y se reinicia el proceso}
\end{itemize}
\section{Creación de vértices}
Existen diversas funciones para crear vertices, dependiendo de si quieren utilizar numeros flotantes o enteros, y si se quieren utilizar tres dimensiones o solo dos. Esas funciones son las siguientes.
\begin{lstlisting}[language=C]
  glVertex2f(GLfloat x, GLfloat y);
  glVertex2i(GLint x, GLint y);
  glVertex3f(GLfloat x, GLfloat y, GLfloat z);
  glVertex3i(GLint x, GLint y, GLint z);
\end{lstlisting}
\section{Finalizacion de la estructura grafica}
Una vez que se han terminado de introducir los vertices, hace falta llamar a la siguiente funcion sin ningun argumento.
\begin{lstlisting}[language=C]
  glEnd();
\end{lstlisting}
