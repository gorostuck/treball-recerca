\chapter{Espacio de vista y transformación de la vista del modelo}
En este paso los objetos son transformados del espacio local al espacio de vista utilizando la matriz \textbf{GL\_MODELVIEW}. Esta matriz es una combinación de las matrices de Modelo y Vista. (\(M_{view} \cdot M_{model}\)). La transformación de Modelo es para convertir del espacio local del objeto al espacio global. Y, la transformación de Vista es para convertir del espacio global al espacio de vista.

\begin{figure}[ht]
  \centering
  \(
  \begin{pmatrix}
    x_{vista}\\y_{vista}\\z_{vista}\\w_{vista}
  \end{pmatrix}
  = M_{modeloVista} \cdot
  \begin{pmatrix}
    x_{obj}\\y_{obj}\\z_{obj}\\w_{obj}
  \end{pmatrix}
  = M_{vista} \cdot M_{modelo} \cdot
  \begin{pmatrix}
    x_{obj}\\y_{obj}\\z_{obj}\\w_{obj}
  \end{pmatrix}
  \)
\end{figure}

Nótese que no hay una matriz de cámara (vista) separada en OpenGL. Entonces, para simular la transformación de la cámara o de la vista, la escena (objetos 3D y luces) deben ser transformados con lo inverso de la transformación de vista. En otras palabras, OpenGL define que la cámara siempre se encuentra en (0, 0, 0) mirando al eje -Z en las coordenadas de espacio de vista, y no puede ser transformada.

\newpage

\section{Transformación de vectores normales}
Los vectores normales también son transformados desde coordenadas locales de objeto a coordenadas de vista para hacer cálculos de iluminación.
Es importante destacar que los vectores normales no son transformados de la misma forma.
\\
Para un punto $(P_x, P_y, P_z)$ en un plano;
Con un vector normal $(n_x, n_y, n_z)$ en el mismo plano;
\\
La ecuación del plano es la siguiente:
\begin{equation*}
  n_xx + n_yy + n_zz  = 0
\end{equation*}
Ahora podemos sustituir con las coordenadas de ese punto y resulta en:
\begin{equation*}
    n_xP_x + n_yP_y + n_zP_z  = 0
\end{equation*}
En forma de producto de matrices:
\begin{equation*}
  \begin{pmatrix}
    n_x && n_y && n_z
  \end{pmatrix}
  \cdot
  \begin{pmatrix}
    P_x \\ P_y \\ P_z
  \end{pmatrix}
  = 0
\end{equation*}

Nótese que para poder resolver la operación correctamente, la matriz del vector normal está transpuesta. De forma más simple, la operación es la siguiente:
\begin{equation*}
  n^t \cdot P = 0
\end{equation*}

La inversa de una matriz por la matriz original es igual a la matriz identidad. Por ello, la matriz inversa de GL\_MODELVIEW por GL\_MODELVIEW es igual a la matriz identidad. Esto se puede introducir en la ecuación y seguiría siendo equivalente.

\begin{equation*}
  \underbrace{n^t \cdot M^{-1}}_{normal} \underbrace{M \cdot P}_{vértice} = 0
\end{equation*}
Con esta ecuación hemos llegado a la forma de calcular la transformación de las normales. \\

\newpage
La forma final de la ecuación se obtiene aplicando la regla $A^t \cdot B^t = (B \cdot A)^t$.
\begin{equation*}
  n^t \cdot M^{-1} = n^t \cdot \Big((M^{-1})^t\Big)^t = \Big((M^{-1})^t\cdot n\Big)^t
  \end{equation*}

La forma en matriz de la operación es la siguiente.

\begin{figure}[ht!]
  \centering
  \(
  \begin{pmatrix}
    nx_{vista}\\ny_{vista}\\nz_{vista}\\nw_{vista}
  \end{pmatrix}
  = ((M_{modeloVista})^{-1})^t \cdot
  \begin{pmatrix}
    nx_{obj}\\ny_{obj}\\nz_{obj}\\nw_{obj}
  \end{pmatrix}
  \)
\end{figure}

\section{Matriz de modelo-vista (GL\_MODELVIEW)}
La matriz GL\_MODELVIEW combina las matrices de vista y de modelado en una sola matriz. Para transformar la vista (cámara), es necesario mover la escena con la transformación inversa.

\begin{figure}[ht]
  \centering
  \(
  \begin{pmatrix}
    \textcolor{red}{m_0} && \textcolor{green}{m_4} && \textcolor{blue}{m_8} && m_{12}\\
    \textcolor{red}{m_1} && \textcolor{green}{m_5} && \textcolor{blue}{m_9} && m_{13}\\
    \textcolor{red}{m_2} && \textcolor{green}{m_6} && \textcolor{blue}{m_{10}} && m_{14}\\
    \textcolor{gray}{m_3} && \textcolor{gray}{m_7} && \textcolor{gray}{m_{11}} && \textcolor{gray}{m_{15}}
  \end{pmatrix}
  \)
\end{figure}

Los 3 elementos más a la derecha de la matriz \((m_{12}, m_{13}, m_{14})\) son para la transformación de translación, \textbf{glTranslatef()}. El elemento \(m_{15}\) es la coordenada homogénia. Se usa específicamente para la transformación proyectiva.

3 conjuntos de elementos, \((m_0, m_1, m_2)\), \((m_4, m_5, m_6)\) y \((m_8, m_9, m_{10})\) son para transformaciones euclidianas y afines, como la rotación \textbf{glRotatef()} o el escalado \textbf{glScalef()}. Nótese que estos 3 sets están representando 3 ejes ortogonales:

\begin{itemize}
\item{\((m_0, m_1, m_2)\) : eje +X, vector \textit{izquierda}, (1, 0, 0) por defecto}
\item{\((m_4, m_5, m_6)\) : eje +Y, vector \textit{arriba}, (0, 1, 0) por defecto}
\item{\((m_8, m_9, m_{10})\) : eje +Z, vector \textit{adelante}, (0, 0, 1) por defecto}
\end{itemize}

\section{Reseteo}
Para restaurar la matriz a su estado por defecto se hace uso de las siguientes funciones. La primera sólo es necesaria en caso de que la matriz seleccionada actualmente no sea GL\_MODELVIEW.
\begin{lstlisting}[language=C]
  glMatrixMode(GL_MODELVIEW);
  glLoadIdentity();
\end{lstlisting}

\section{Translación}
Una operación de translación consiste en desplazar un punto por un vector de coordenadas (x, y, z). La operación tiene el siguiente aspecto:

\begin{minipage}{0.3\textwidth}
  \centering
  \begin{tikzpicture}[thick ,scale=0.6, every node/.style={scale=0.9}, line cap=round,line join=round,>=triangle 45,x=1cm,y=1cm]
    \clip(-7,-2.4) rectangle (-1.5,1.5);
    \draw [->,>=stealth,line width=0.5pt,color=gray] (-5.22,-1.72) -- (-3.32,0.24);
    \begin{scriptsize}
      \draw [fill=black] (-5.22,-1.72) circle (1pt);
      \draw[color=black] (-5.142466613438916,-2) node {(x, y, z)};
      \draw [fill=black] (-3.32,0.24) circle (1pt);
      \draw[color=black] (-3.2574601569280657,0.56) node {(x+X, y+Y, z+Z)};
      \draw[color=gray] (-3,-0.8810231214816775) node {(X, Y, Z)};
    \end{scriptsize}
  \end{tikzpicture}
\end{minipage}\begin{minipage}{0.5\textwidth}
  \centering
  \[
  \begin{bmatrix}
    1 && 0 && 0 && X\\
    0 && 1 && 0 && Y\\
    0 && 0 && 1 && Z\\
    0 && 0 && 0 && 1
  \end{bmatrix}
  \cdot
  \begin{pmatrix}
    x \\ y \\ z \\ 1
  \end{pmatrix}
  =
  \begin{pmatrix}
    x + X \cdot 1 \\
    y + Y \cdot 1 \\
    z + Z \cdot 1 \\
    1
  \end{pmatrix}
  \]
\end{minipage}

Con la operación de translación la matriz actual se traslada por (x, y, z). Primero, se crea una matriz de translación, $M_T$, después se multiplica con la matriz seleccionada actualmente para producir la matriz de transformación final: $M = M \cdot M_T$
\begin{figure} [h!]
  \centering
  \(
  \begin{bmatrix}
    1 && 0 && 0 && x\\
    0 && 1 && 0 && y\\
    0 && 0 && 1 && z\\
    0 && 0 && 0 && 1
  \end{bmatrix}
  \)
  \caption{Matriz de translación, $M_T$}
\end{figure}

La función utilizada en OpenGL para efectuar una transformación es la siguiente.
\begin{lstlisting}[language=C]
  glTranslatef(GLfloat x, GLfloat y, GLfloat z);
\end{lstlisting}
En los argumentos pasamos los valores de las coordenadas del vector deseado.

\newpage
\section{Rotación}
Una transformación de rotación rota un vector alrededor del origen (0, 0, 0) utilizando un eje y un ángulo.
\begin{figure} [ht!]
  \centering
  \begin{tikzpicture}[thick, scale=0.6, every node/.style={scale=1.2}, line cap=round, line join=round, >=triangle 45, x=1cm, y=1cm]
    \clip(-2, 0) rectangle (2, 2);
    \draw [->, >=stealth, line width=1pt, color=red] (0,0) -- (2, 2);
    \draw [->, >=stealth, line width=1pt, color=blue] (0,0) -- (-2, 2);
    \draw (0.702,0.702) arc[radius=1cm, start angle=45, end angle=(90+45)];
    
    \begin{scriptsize}
      \draw[color=black] (0, 1.2) node {$\alpha$};
    \end{scriptsize}
    
  \end{tikzpicture}
\end{figure}

Los objetos pueden ser rotados alrededor de cualquier eje, pero por ahora, sólo son importantes los ejes X, Y y Z. Posteriormente se explicará que se puede establecer cualquier eje de rotación rotando por el eje X, Y y Z simultáneamente.
Para comprender de donde salen las matrices de rotación se puede considerar el siguiente ejemplo. Imaginemos que vamos a rotar alrededor del eje Z, esto lleva a que los únicos valores que serán alterados serán los de las coordenadas X y Y. Vamos a ver cómo se rotan por el ángulo $\alpha$ los siguientes vectores.

\begin{minipage}{0.5\textwidth}
  \centering
  \begin{tikzpicture}[thick, scale=0.6, every node/.style={scale=1.1}, line cap=round, line join=round, >=triangle 45, x=1cm, y=1cm]
    \clip (-0.5, -0.5) rectangle (6, 6);
    \draw [->, >= stealth, line width=0.5pt, color=black] (0,0) -- (0, 5);
    \draw [->, >= stealth, line width=0.5pt, color=black] (0,0) -- (5, 0);
    \draw [->, >= stealth, line width=1pt, color=red] (0, 0) -- (3, 0);
    \draw [->, >= stealth, line width=1pt, color=blue] (0, 0) -- (2.598, 1.5);
    \draw (1.5 , 0) arc [radius =1.5cm, start angle=0, end angle=30];
    \begin{scriptsize}
      \draw[color=black] (1, 0.25) node {$\alpha$};
      \draw[color=red] (3, 0.5) node {(1, 0)};
      \draw[color=blue] (3, 2) node {($cos \alpha$, $sin \alpha$)};
    \end{scriptsize}
    \end{tikzpicture}
\end{minipage}
\begin{minipage}{0.5\textwidth}
  \centering
    \begin{tikzpicture}[thick, scale=0.6, every node/.style={scale=1.1}, line cap=round, line join=round, >=triangle 45, x=1cm, y=1cm]
    \clip (-6, -0.5) rectangle (6, 6);
    \draw [->, >= stealth, line width=0.5pt, color=black] (0,0) -- (0, 5);
    \draw [->, >= stealth, line width=0.5pt, color=black] (0,0) -- (5, 0);
    \draw [->, >= stealth, line width=0.5pt, color=black] (0,0) -- (-5, 0);
    \draw [->, >= stealth, line width=1pt, color=red] (0, 0) -- (0, 3);
    \draw [->, >= stealth, line width=1pt, color=blue] (0, 0) -- (-1.5, 2.598);
    \draw (0 , 1.5) arc [radius =1.5cm, start angle=90, end angle=120];
    \begin{scriptsize}
      \draw[color=black] (-0.25, 1) node {$\alpha$};
      \draw[color=red] (1, 2.5) node {(0, 1)};
      \draw[color=blue] (-2.7, 1.5) node {($-sin \alpha$, $cos \alpha$)};
    \end{scriptsize}
    \end{tikzpicture}
    
\end{minipage}

\newpage

Por lo tanto, la operación con matrices para rotar alrededor del eje Z con un ángulo $\alpha$ es la siguiente:
\begin{equation*}
  \begin{bmatrix}
    cos \alpha && -sin\alpha && 0 && 0 \\
    sin \alpha && cos\alpha && 0 && 0 \\
    0 && 0 && 1 && 0 \\
    0 && 0 && 0 && 1
  \end{bmatrix}
  \cdot
  \begin{pmatrix}
    x \\ y \\ z \\ 1
  \end{pmatrix}
  =
  \begin{pmatrix}
    cos \alpha \cdot x - sin \alpha \cdot y \\
    sin \alpha \cdot x + cos \alpha \cdot y \\
    z \\
    1
  \end{pmatrix}
\end{equation*}

Una operación similar ocurre en la rotación alrededor del eje X:
\begin{equation*}
  \begin{bmatrix}
    1 && 0 && 0 && 0 \\
    0 && cos \alpha && -sin \alpha && 0 \\
    0 && sin \alpha && cos \alpha && 0 \\
    0 && 0 && 0 && 1
  \end{bmatrix}
  \cdot
  \begin{pmatrix}
    x \\ y \\ z \\ 1
  \end{pmatrix}
  =
  \begin{pmatrix}
    x \\
    cos \alpha \cdot y - sin \alpha \cdot z \\
    sin \alpha \cdot y + cos \alpha \cdot z \\
    1
  \end{pmatrix}
\end{equation*}

Y por último, la rotación alrededor del eje Y:
\begin{equation*}
  \begin{bmatrix}
    cos \alpha && 0 && sin \alpha && 0 \\
    0 && 1 && 0 && 0 \\
    - sin \alpha && 0 && cos \alpha && 0 \\
    0 && 0 && 0 && 1
  \end{bmatrix}
  \cdot
  \begin{pmatrix}
    x \\ y \\ z \\ 1
  \end{pmatrix}
  =
  \begin{pmatrix}
    cos \alpha \cdot x + sin \alpha \cdot z \\
    y \\
    - sin \alpha \cdot x + cos \alpha \cdot z \\
    1
  \end{pmatrix}
\end{equation*}


Estas tres matrices se pueden combinar en una misma con parámetros (x, y, z) para definir el eje de rotación y un ángulo $\alpha$ por el cuál rotar. Esta matriz de rotación $M_R$ luego es multiplicada por la matriz seleccionada actualmente y sustituye esa por el resultado: $M = M \cdot M_R$.
\begin{figure} [ht]
  \centering
  \(
  \begin{pmatrix}
    x^2(1-c) + c  && xy(1-c) - zs && xz(1-c) + ys && 0\\
    xy(1-c) + zs && y^2(1-c) + c  && yz(1-c) - xs && 0\\
    xz(1-c) - ys && yz(1-c) + xs && z^2(1-c) + c  && 0\\
    0 && 0 && 0 && 1
  \end{pmatrix}
  \)
  \\donde $c = cos \alpha$, $s = sin \alpha$ 
  \caption{Matriz de rotación, $M_R$}
\end{figure}

La función utilizada en OpenGL para efectuar una rotación es la siguiente.
\begin{lstlisting}[language=C]
  glRotatef(GLfloat angle, GLfloat x, GLfloat y, GLfloat z);
\end{lstlisting}
En los argumentos pasamos el ángulo (en grados) que se usará cómo $\alpha$ en la matriz de rotación y los valores de los ejes.

\newpage
\section{Escalado}
Una transformación de escalado consiste en escalar cada componente de un vector por un número (que puede ser diferente). Se puede utilizar para alargar o acortar un vector. La operación es así:

\begin{minipage}{0.3\textwidth}
  \centering
  \begin{tikzpicture}[thick ,scale=0.6, every node/.style={scale=0.9}, line cap=round,line join=round,>=triangle 45,x=1cm,y=1cm]
    \clip(-3, -3) rectangle (3, 3);
    \draw [->,>=stealth, line width=0.5pt, color=blue] (-2,-2) -- (2,2);
    \draw [->,>=stealth, line width=0.5pt, color=red] (-2,-2) -- (0,0);
    \begin{scriptsize}
      \draw[color=red] (0,-1) node {(x, y, z)};
      \draw[color=blue] (2,0.8) node {(2x, 2y, z)};
      \draw[color=black] (1,-2) node {Escalar por (2,2,1)};
    \end{scriptsize}
  \end{tikzpicture}
\end{minipage}
\begin{minipage}{0.5\textwidth}
  \centering
  \[
  \begin{bmatrix}
    SX && 0 && 0 && 0\\
    0 && SY && 0 && 0\\
    0 && 0 && SZ && 0\\
    0 && 0 && 0 && 1
  \end{bmatrix}
  \cdot
  \begin{pmatrix}
    x \\ y \\ z \\ 1
  \end{pmatrix}
  =
  \begin{pmatrix}
    SX \cdot x \\
    SY \cdot 1 \\
    SZ \cdot 1 \\
    1
  \end{pmatrix}
  \]
\end{minipage}


El escalado produce una matriz de transformación de escalado no uniforme en cada eje (x, y, z) multiplicando la matriz de escalado por la matriz seleccionada actualmente. $M = M \cdot M_S$
\begin{figure} [ht]
  \centering
  \(
  \begin{pmatrix}
    x && 0 && 0 && 0\\
    0 && y && 0 && 0\\
    0 && 0 && z && 0\\
    0 && 0 && 0 && 1
  \end{pmatrix}
  \)
  \caption{Matriz de escalado, $M_S$}
\end{figure}

La función utilizada en OpenGL para efectuar un escalado es la siguiente.
\begin{lstlisting}[language=C]
  glScalef(GLfloat x, GLfloat y, GLfloat z);
\end{lstlisting}
En los tres parámetros se indica el factor de escalado deseado en cada uno de los ejes.

