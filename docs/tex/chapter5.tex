\chapter{Transformación de la vista del modelo}
\section{Matriz de transformación}
La matriz de transformación es aquella que se aplica a los diferentes vértices cuando son llamados y que acumula operaciones de \textbf{Translación}, \textbf{Rotación} y \textbf{Escalado}.
 \begin{figure}[ht]
  \centering
  \begin{bmatrix}
    1 && 0 && 0 && x\\
    0 && 1 && 0 && y\\
    0 && 0 && 1 && z\\
    0 && 0 && 0 && 1
  \end{bmatrix}
 \end{figure}
 \subsection{Operación: Reseteo}
Para restaurar la matriz a su estado original se hace uso de la función
\begin{lstlisting}[language=C]
  glLoadIdentity();
\end{lstlisting}
\subsection{Operación: Traslación}
Con la operación de Traslación se desplaza la matriz de transformación por un vector llamando a la función
\begin{lstlisting}[language=C]
  glTransformf(GLfloat x, GLfloat y, GLfloat z);
\end{lstlisting}
En los argumentos ponemos los valores del vector deseado.
\subsubsection{Operación matemática}
\(\overrightarrow{V}}\) es el vector que hemos pasado como argumento, \(\mathbf{transf}\) es la matriz de transformación.
\begin{equation}
  \begin{bmatrix}
    1 && 0 && 0 && \overrightarrow{V}_x\\
    0 && 1 && 0 && \overrightarrow{V}_y\\
    0 && 0 && 1 && \overrightarrow{V}_z\\
    0 && 0 && 0 && 1
  \end{bmatrix}
  \cdot
  \begin{bmatrix}
    \mathbf{transf}_x\\\mathbf{transf}_y\\\mathbf{transf}_z\\1
  \end{bmatrix} =
  \begin{bmatrix}
    \mathbf{transf}_x'\\\mathbf{transf}_y'\\\mathbf{transf}_y'\\1
  \end{bmatrix}
\end{equation}
