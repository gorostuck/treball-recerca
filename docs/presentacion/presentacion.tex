\documentclass{beamer}
\mode<presentation>
    {
      \usetheme{Madrid}
      \usecolortheme{default}
      \usefonttheme{default}
      \setbeamertemplate{navigation symbols}{}
      \setbeamertemplate{caption}[numbered]
    }

    
\usepackage[english]{babel}
\usepackage[utf8x]{inputenc}

\title[T. e I. de un Motor Gráfico]{Teoría e Implementación de un Motor Gráfico}
\author{Joel Pérez Ferrer}
\institute{Institut de Bruguers}
\date{3/11/2017}

\begin{document}

\begin{frame}
  \titlepage
\end{frame}

\section{Introducción}
\begin{frame}{¿Qué es un motor gráfico?}
\end{frame}
\begin{frame}{Objetivos del Treball de Recerca}
\end{frame}
\begin{frame}{Hipótesis}
\end{frame}
\begin{frame}{Metodología}
\end{frame}


\section{Teoría de un motor gráfico}
\begin{frame}{Espacio local}
\end{frame}
\begin{frame}{Espacio global}
\end{frame}
\begin{frame}{Espacio de vista}
\end{frame}
\begin{frame}{Espacio NDC}
\end{frame}
\begin{frame}{Espacio de ventana}
\end{frame}

\section{Conclusión}
\begin{frame}{Ajustes del método}
\end{frame}
\begin{frame}{Revisión de la hipótesis}
\end{frame}
\begin{frame}{Preguntas}
\end{frame}
\begin{frame}{Valoración personal}
\end{frame}

\end{document}
