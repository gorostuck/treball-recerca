\documentclass{report}

\usepackage[utf8]{inputenc}
\usepackage{graphicx}
\usepackage{listings}
\usepackage{mathtools}

\usepackage{pgf,tikz}\usepackage{mathrsfs}\usetikzlibrary{arrows}


\renewcommand{\contentsname}{Contenidos}
\renewcommand{\chaptername}{Capítulo}

\title{Esborrany Treball de Recerca:\\Teoría e implementación de un motor gráfico}
\date{Curso 2017/2018}
\author{Alumno: Joel Pérez\\Profesor: José Ramírez\\Departamento de Matemáticas}


\begin{document}
\lstset{language=C, basicstyle=\ttfamily}
\pagenumbering{gobble}

\maketitle


\newpage

\pagenumbering {arabic}

\begin{abstract}
  En este Treball de Recerca he intentado demostrar cómo funcionan las principales librerías gráficas comerciales y poder entender los diversos cálculos que se efectuan en la ejecución de una aplicación gráfica 3D. Para esto he programado una librería gráfica relativamente simple y la he usado en un motor gráfico que también he programado.
\end{abstract}

\clearpage
\phantomsection
\tableofcontents

\newpage
\chapter{Introducción}
En este capítulo se hará una introducción al Treball de Recerca de forma similar a la que se hizo en el esborrany.
\section{Objectivos}
\section{Motivación personal}
\section{Relevancia del estudio en el campo}
\section{Límites del trabajo}
\section{Metodología empleada para realizar la investigación}
\subsection{Programación}
\subsection{Escritura del documento}
\section{Hipótesis}
\newpage

\chapter{Conceptos previos}
Previamente a la lectura del documento se familiarizará al lector con conceptos que considero necesarios para la comprensión del mismo. Sean tanto palabras técnicas de el mundo de la programación como conceptos matemáticos generales.
\newpage

\chapter{Teoría e implementación de la librería gráfica}
En este capítulo se comentarán todos los aspectos de las funciones creadas en la librería gráfica. Primero desde un punto de vista teórico y después se observará la implementación. Además se comparará con el funcionamiento de una librería gráfica popular, OpenGL.
\newpage
\chapter{Teoría e implementación de el motor gráfico}
Este capítulo está dedicado a observar desde un punto de vista primero teórico y luego práctico el uso de la librería gráfica una vez aplicado en un motor gráfico.
\newpage
\chapter{Situación actual}
En este capítulo mi objetivo es recopilar información extraída de individuos familiarizados con el desarrollo de software pero por motivos diferentes y intentar exponer su opinión en el software hecho a mano.
\newpage
\chapter{Valoración general y conclusión}
Finalmente, utilizando las nociones aprendidas en el transcurso de este Treball de Recerca intentaré llegar a una conclusión y ver si la hipótesis presentada realmente se sostiene tras finalizar el trabajo.
\end{document}

